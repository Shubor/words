%========================================
% Info2120
%========================================

\documentclass[10pt, multicolumn, a4paper]{article}

\usepackage{amsmath} % just math
\usepackage{amssymb} % allow blackboard bold (aka N,R,Q sets)
\usepackage{ulem}
\usepackage{enumitem}
\usepackage{tikz}
\setlist{nolistsep}
\usepackage{multicol}
\usepackage[left=2cm,top=1cm,right=2cm,nohead,nofoot]{geometry}

\iffalse
\setlength{\parskip}{0pt}
\setlength{\parsep}{0pt}
\setlength{\headsep}{0pt}
\setlength{\topskip}{0pt}
\setlength{\topmargin}{0pt}
\setlength{\topsep}{0pt}
\setlength{\partopsep}{0pt}
\linespread{0.5}

\usepackage[compact]{titlesec}
\titlespacing{\section}{0pt}{*0}{*0}
\titlespacing{\subsection}{0pt}{*0}{*0}
\titlespacing{\subsubsection}{0pt}{*0}{*0}
\fi

\newcommand{\lecture}[1]
{
	\hrule
	\section{#1}
	\hrule
}

\begin{document}

\linespread{1} % single spaces lines
\small \normalsize %% dumb, but have to do this for the prev to work

%========================================
% Lecture 1
%========================================

\lecture{Introduction to DB Management Systems}

\begin{multicols}{2}

\iffalse % --------- Outline ----------------
	File Systems vs. DBMS
	Overview of Core Database Functionalities
		- Data Independence
		- Declarative Querying
		- Transactions
	Metadata
\fi % ------------------------------------
	
	\subsection*{Database}
	\begin{itemize}
	\item collection of data
	\item persistent
	\item shared: qualified users have access to the same data
	\end{itemize}
	
	\subsection*{DB Management System}
	\begin{itemize}
	\item data independence
	\item declarative querying
	\item transaction management and concurrency control 
	\end{itemize}

	\subsection*{File System vs. DBMS}
	\begin{itemize}
	\item data storage
		\begin{itemize}
		\item FS: repeat data definitions 
		\item DBMS: keeps metadata about its content and state
		\end{itemize}
	\item data access
		\begin{itemize}
		\item FS: program repeatedly
		\item DBMS: declarative queries
		\end{itemize}
	\end{itemize}

\end{multicols}

%========================================
% Lecture 2
%========================================

\lecture{Conceptual Database Design: The Relational Model}

\begin{multicols}{2}

\iffalse % --------- Outline ----------------
	Conceptual Database Design using the:
		- Entity Relationship Model
		- Database Design with UML
\fi % ------------------------------------

	\subsection*{Entity Relationship Model}
	\begin{itemize}
	\item data model that depicts associations between different categories of data
	\item can be converted into a relational schema
	\end{itemize}

\end{multicols}

%========================================
% Lecture 3
%========================================

\iffalse
\lecture{}

\iffalse % --------- Outline ----------------
	
\fi % ------------------------------------

\begin{multicols}{2}

\end{multicols}
\fi

%========================================
% Definitions
%========================================

\hrule
\section*{Definitions} 
\hrule

\begin{multicols}{2}
	
	\begin{description}
	\item[data] stored representation of raw facts
	\item[information] processed data
	\item[metadata] data that describes data
	\item[relational Database] collection of tables of particular `types', each of which store data as rows with multiple attributes
	\iffalse
	\item[data model] concepts for describing data: structure, operations and constraints
	\item[schema] a description of a particular data collection at abstraction level using given model
	\item[relational data model] a relation is a table with rows and columns; described by a schema
	\fi
	\end{description}

	
\end{multicols}

%========================================
%========================================

\end{document}